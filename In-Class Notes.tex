\documentclass{article}
\usepackage{amsmath,amsthm,mathrsfs,amssymb,graphicx,bm,hyperref,url}
\usepackage{geometry}
\title{ASTRO 2 In-Class Notes}
\author{Bill Wolf}
\date{Updated \today}
\begin{document}
\maketitle
\newpage{}
\tableofcontents
\newpage{}
		\textit{March 29, 2011}	
	\section{Introduction to Cosmology: Where Did It All Come From?}
	This question has bothered mankind since prehistoric times. Why is there something instead of nothing? How did the universe come to be as it is now? The scientific pursuit of the answers of these questions is the study of the field of Cosmology. Over time, scientists have developed the tools of rational thought and science to help us answer these deep questions.
	\subsection{More specific Questions}
	\begin{itemize}
		\item Is the Universe evolving?
		\item If so, how and when did it form?
		\item How big/old is the universe?
		\item How/when did galaxies/stars/planets form?
		\item Is there extraterrestrial life in the Universe?
	\end{itemize}
	\subsection{Physical Cosmology}
	Cosmology is not an isolated science. Indeed, we use all the tools of physics to probe the nature of the universe. These tools are being used to probe even the most unknown topics in Cosmology, including the mysterious dark matter and dark energy. However, cosmology is unique in that we only have one universe to study. We cannot have a ``control'' universe and other universes to play around with in experiments. We are terribly limited by our position in space and time to study the universe. All data we gather is gathered from the here and now, on Earth and during our life time, which is very short on astronomical scales.
	\subsection{Telescopes as Time Machines}
	Light travels at a finite speed, so light that arrives on Earth was emitted from its source some time ago. Thus, the further away an object is, the further back in time we are looking when we view the object in a telescope. That being said, objects can only be seen at a \emph{specific} time in their life time. We must then look at many galaxies at different points in their lives to figure out how galaxies and the universe have evolved in time. This is not unlike reconstructing a family history from a series of photographs taken at different times.\\
	
	\noindent This is done by using telescopes that probe in all different wavelengths, not only the visible. Right now we are in a golden age of telescope varieties. The Chandra telescope is probing the sky for radiation in the X-Ray spectrum. The Spitzer Space Telescope is probing in the infrared, and Hubble is probing in the visible. In the near future, the James Webb telescope will be probing in an even wider range of wavelengths.
	
	\subsection{The Evolution of Cosmology}
	Cosmology is an ever-changing field of study. The ancient Greeks once believed that the Earth was at the center of the universe. However, observations have now made that theory difficult to support. Every year new discoveries similarly challenge and change our view on the universe. In 1929, Edwin Hubble proved that the universe was expanding, which was before an unheard of idea. In more recent years, this expansion was found to be \emph{accelerating}, giving rise to dark energy. This notion is still not well-understood, and is analogous to throwing a ball in the air and having it accelerate upwards out of Earth's gravitational field.\\
	
	\noindent Galileo's observations of sunspots proved that the heavens change in time. Hubble showed that the universe is not static. Measures of gravitational accelerations have shown that there is a massive form of matter that does not interact with regular matter (dark matter). Our view of the universe is in a constant state of flux.
	
	\subsection{The Dilemma of Cosmology and Cosmological Principles} 
	We are only able to make measurements from Earth... this is not an ideal situation from a scientific point of view. We'd like to change views to see what changes and what doesn't. This problem is ``solved'' by the Copernican Principle.
	
	\subsubsection{The Copernican Principle} We assume that Earth is not ``special'' in any particular way with regards to its position in the universe. In other words, our local sample of the universe is no different from more remote and inaccessible places. This is related to two more fundamental principles in physics:
	\begin{itemize}
		\item \textbf{Relativity:} The laws of physics (whatever they are) do not depend of space and time.
		\item \textbf{Occam's Razor:} Physical explanations of natural phenomena should be as simple as possible.
	\end{itemize}
	These are not proven by any means, but they seem a reasonable assumption since they have not been effectively disproven to date.
	\subsubsection{Homogeneity and Isotropy}
	An alternative way to state this principle is to assume that the universe is homogeneous and isotropic. That is, the universe is the same in any direction and in any location. This means that there is no ``special'' place in the universe. As has been shown in simulations, the universe appears to form a \textbf{cosmic web}. Matter and energy chunk together in filaments and form voids. So, on large enough scales, the universe looks the same in every place, but if we zoom in, filaments and voids will obviously be different. Thus, homogeneity and isotropy are valid in the \emph{large-scale limit}.
	\subsubsection{Perfect Cosmological Principle} An outdated theory was the perfect cosmological principle. This principle assumed that the universe was homogeneous, isotropic, and \emph{time-invariant}. However, observations have shown that the universe has indeed changed throughout time, so this is no longer a valid assumption in Cosmology.
	\subsubsection{Anthropic Cosmological Principle} The anthropic cosmological principle states that the fact that we exist to observe the universe makes now a special point in time. This more philosophical argument is sort of like quantum mechanics on steroids and can be used to explain some properties of the universe, but it is not a very accepted argument in physics.\\
	
	\noindent\textit{March 31, 2011}
	\section{Galaxies}
	\subsection{The Size of the Universe}
	\subsubsection{Our Place in the Universe} We reside in the Milky Way galaxy. Many people still today mix up the ``Universe'' with ''The Milky Way.'' However, the observable universe is strictly larger, as it contains many \emph{many} galaxies. In the past, though, this was not always clear.
	\subsubsection{From Nebulae to Galaxies}
	In older times, anything we now know to be a galaxy was simply regarded as a \textbf{nebula}, which comes from the Latin for cloud. Many of these were catalogued by the French astronomer Messier and put in his famous Messier catalogue, which is a list of things that aren't comets. It was actually a great debate as to how large these ``nebulae'' even were. In the 1920's, a debate between Harlow Shapley and Heber Curtis erupted as to the precise nature of nebulae. Shapley argued that they resided within our own galaxy (galactic) whereas Curtis argued that they were outside of our galaxy, as little island universes (extragalactic). Ultimately this argument went nowhere since simply looking at an object tells you nothing about its size or distance (without knowing either).\\
	
	\noindent It was left to Edwin Hubble to solve this problem. Hubble discovered Cephid variables in M31. Since Cephid variables are standard candles (we \emph{know} what its true luminosity is), he was able to calculate the distance to M31 by using flux arguments:
	$$F=\frac{L}{4\pi d^2}$$
	\paragraph{Example} A Cephid in IC4182 has a period of 42 days and an apparent magnitude of $m=22.0$ in the V band. From the period luminosity relation of Cephid variables,, we know that the absolute luminosity is $M=-6.5$ in the V band. If we recall the relationship between the apparent and absolute magnitude:
	$$m=M+5\log(d/pc)-5$$
	Inverting this equation, we get
	$$d=10^{0.2(m_M+5)}\,\mathrm{pc}$$
	Son in this case, $d=5\times10^6\,\mathrm{pc}$.\hfill $\square$\\
	
	\noindent Using this technique, Hubble discovered that M31 is $70\,\mathrm{kpc}$ across (comparable to the Milky Way) and 150 kpc away from us. Now we know that there are billions of such galaxies. In fact, in just one square degree (about the size of the moon in the sky), there are tens of thousands of visible galaxies. One need only point a large enough telescope in that direction. Overnight, people realized that the universe was many times large than they had ever thought.
	\subsubsection{An Idea of the Immense Size of the Universe}
	The universe is \emph{much} bigger than the Milky Way, containing billions of galaxies, each one tens of kpc in size. In fact the visible universe is of order of gpc, or gigaparsecs, equivalent to \emph{millions} of times that of the Milky Way.
	\subsection{Types of Galaxies}
	Hubble classified galaxies under a fairly general scheme. The two largest classes of galaxies are spirals and ellipticals. The spiral galaxies are further divided up into barred spirals and regular spirals. Unclassifiable galaxies are called irregular. Classifying galaxies by shape is called analyzing their \textbf{morphology}.
	\subsubsection{Elliptic Galaxies}
	Elliptic Galaxies are classified by how elliptic they appear to us in the sky. The roundest elliptical galaxies are called E0 galaxies. More elliptic galaxies (like E6) are more elliptic. The number is given by
	$$n=\left(1-\frac{b}{a}\right)\times 10$$
	where $a$ and $b$ are the semi-major and -minor axes of the apparent ellipse in the sky.\\
	
	\noindent
	It's also helpful to look at the spectrum of a galaxy to identify what elements are present in it as well as how hot or cold the constituent materials are. Elliptic galaxies emit a lot of redder light, implying that they are composed primarily of older stars. Additionally, elliptical galaxies have very little gas and dust in them.
	\subsubsection{Spiral Galaxies}
	Spiral galaxies have a central bulge and then spiral arms coming out from the center. Additionally, many (about $3/4$) develop instabilities which manifest themselves as bars from the center. Spiral galaxies exhibit emission lines and have bluer light. This tells us that there are many young stars as well as interstellar gas and dust. 
	\subsubsection{Lenticular Galaxies}
	Lenticular (S0) galaxies are more or less a ``transition'' class of galaxies between elliptical and spiral galaxies. They exhibit a central bulge, but they don't exhibit spiral arms. Instead, they are generally rotationally symmetric about their main axis. Popular theories of lenticular galaxies state that they may just be old, tired spiral galaxies.
	\subsubsection{Summary of Galaxy Types}
	Galaxies are not free to take on any shape they wish. They generally fall into one of four classes: spiral, elliptic, lenticular, or irregular, with Elliptic galaxies having older stars and spiral galaxies containing younger stars.
	\subsection{Structure of Galaxies}
	 \subsubsection{Rotational Support} The shapes of the galaxies are supported by rotation in spiral galaxies and pressure in elliptical galaxies. We can use the rotation rate to calculate a galaxy's mass via
	 $$\frac{GM}{R^2}=\frac{v^2}{R}$$
	 so $$M(R)=\frac{v^2 R}{G}$$
	 Where $M(R)$ is the mass held within a radius $R$. We are often able to measure the speeds of stars in a galaxy via their doppler shifts, and then we can calculate the distance from the center via the small angle formula and other nifty tools.
	 \subsubsection{Pressure Support}
	 It has been found that Elliptical Galaxies are not rotating fast enough for rotational support alone to hold their shape. Instead, the pressure of the galaxy holds it in its shape, much like the air pressure in a balloon keeps it from collapsing. However, its not the gas and dust (of which there is very little) that is causing the pressure, but the actual movement of the stars.\\
	 
	 \noindent From a spectrum, we can measure the pressure, $\sigma$, and from the distance, we infer the size r$R$. Then using the Virial Theorem, we obtain
	 $$M=\frac{k\sigma^2R}{G}$$
	 where $k$ is a constant of order unity that varies with the nature of the galaxy in question.
	 \subsection{Type Ia Supernovae}
	 \subsubsection{Standard Candles}
	 If we know the intrinsic luminosity of an object and we measure the apparent flux, we can obtain the distance to that object. Thankfully, there are certain objects whose luminosity we know, no matter where they are. These objects are called ``standard candles.'' We have already seen one very useful variety of standard candles: Cephid Variables. These help us determine distances out to a certain distance, but we need more powerful tools to prober further out.
	 \subsubsection{Conditions for a Type Ia Supernova}
	 When a binary system consists of a white dwarf and a large companion star who has overvilled its Roche Lobe, material from the companion star flows onto the surface of the white dwarf. Eventually, the mass gets too high and the white dwarf exploades, causing a Type Ia supernova, which is a known brightness.\\
	 
	 \noindent \textit{April 7, 2011}
	 \subsection{Large-Scale Structure}
	 Galaxies are not isolated. Instead, they tend to exist in groups, which we call\ldots groups. For instance, the Milky Way is part of ``The Local Group.'' Larger scale collections of galaxies are called clusters, or superclusters. This is a product of the tendency of matter to collect in clumps and filaments. Even a map of the sky shows galaxies existing prevalently on filaments that make up the cosmic web.
	 \subsubsection{Morphology-Density Relationship}
	 More often, very dense clusters typically have older galaxies. Thus, they are often yellower in appearance and consist largely of elliptical and lenticular galaxies. This relationship has been studied as a function of redshift, and it suggests that in the past, spiral galaxies were more prevalent. It thus seems likely that younger galaxies are more typically spiral galaxies, which then combine over time to form inert elliptical galaxies.
	 \subsubsection{Sizes of Clusters}
	 Clusters are inherently gravitationally bound systems, but they do not rotate. Thus, they keep from collapsing by pressure, much like how elliptical galaxies keep from collapsing. And much like elliptical galaxies, we can measure the mass of a cluster via the Virial Theorem:
	 $$M=k\sigma^2 R/G$$
	 \subsection{Galactic Collisions}
	 Cosmic collisions happen on timescales of billions of years. Thus, we cannot understand how galaxies collide by studying one collision. Rather, we study many different galactic mergers all at different stages of colliding.
	 \subsubsection{Effects of Mergers}
	  One common feature of collisions is \textbf{tidal tails}, which are parts of the galaxies that are less affected by gravity and are thus dragged behind each constituent galaxy. Additionally, a \textbf{starburst} effect may occur, where the centers of the galaxies become overly dense, causing a sharp increase in star formation. Collisions can also cause dramatic changes in morphology, i.e. spirals merging into an elliptical galaxy.\\
	  \subsubsection{Changes in Morphology} 
	  Collisions can also cause dramatic changes in morphology, i.e. spirals merging into an elliptical galaxy. It is presently thought that almost all elliptical galaxies were formed from the merger of spirals.\\
	  
	  \noindent The field of galaxy mergers is still a very active area of research.

	 \subsection{Dark Matter}
	 We can determine the mass of a galaxy by observing the orbits of the constituent particles around the center of a galaxy by using Kepler's Third Law. Current measurements show that there is far more mass in galaxies than can be accounted for by ordinary means. There are two solutions to this problem:
	 \begin{enumerate}
	 	\item\textbf{Dark Matter}: There is an extended halo of dark matter that we cannot detect by any means other than by noticing its gravitational effect
		\item\textbf{Gravity Problem}: Perhaps our current understanding of the gravitational force is flawed.
	\end{enumerate}
	\subsubsection{Crash Course on Particle Physics}
		Most of the mass we encounter in our daily life is due to quarks, one of the fundamental particles. Most of matter is made of \textbf{baryons} (heavy particles like protons and neutrons) and \textbf{leptons} (lighter particles like electrons and neutrinos).
	\subsubsection{Dark Matter in Clusters}
		Most of the baryonic mass in clusters is hot gas. This gas does not have time to cool down on relevant time scales. We are able to then measure the presence of this gas via telescopes (Chandra, for example), since it emits X-Rays. However, baryonic mass only accounts for a very small fraction of the total mass of a cluster. One topic of discussion that kept Astronomers busy for a very long time is whether or not dark matter is baryonic.
	\subsubsection{Constituency of the Universe}
	73\% of the energy of the universe is mysterious \textbf{dark energy}. Another 23\% is made of cold dark matter. The reaming 4\% is atoms.\\
	
	\noindent \textit{April 12, 2011}\\
	
	\subsubsection{Evidence: Gravitational Lensing}
	Recall that gravity is not really a force in the strictest sense. Rather, mass bends space around it. This in turn causes light to be curved around massive objects. This can cause the light from objects in the sky to appear to arrive from incorrect positions. Obviously the more massive an object is, the more it will bend light. This phenomenon is known as \textbf{Gravitational Lensing}.\\
	
	\paragraph{Strong Lensing} Sometimes, this effect is so strong that a galaxy that is actually behind another galaxy will appear multiple times around the nearer galaxy. Effective the light has been bent around the galaxy via multiple different paths. This is called \textbf{Strong Lensing}. This is not at all unlike the effects of the stem of a wine glass on a source behind it. In that case, though, the differing index of refraction of the glass causes the refraction, not gravity.\\
	
	\noindent Given a set of images from gravitational lensing, astronomers can measure the average image separation (average separation between mirror images). This, in turn, can be used to estimate the mass of the ``lens galaxy.'' As it turns out, the mass of matter in the galaxy that we can detect alone cannot account for the degree of lensing that we see. This is further evidence for the existence of dark matter.
	
	\paragraph{Weak Lensing} Suppose we are viewing a distant galaxy with no other galaxies in the way. There will still be some \textbf{Weak Lensing} due to ambient gravitational effect. This is manifested through shearing and slight bending of images. This tends to make a group of galaxies appear sheared tangentially, often forming ring-like structure, which tells us that there is some mass in the middle of the ``ring'' that is lensing the image.
	\subsubsection{The effect of Dark Matter and Gravitational Lensing on the Hubble Constant}
	Gravitational lensing is also a useful tool in measuring the size and age of the universe. This is done by using the fact that strong gravitational fields tend to ``slow`` down time. Thus, as light travels close to a lens, it takes longer than it normally would.\\
	
	\noindent Sometimes with multiple images, we can see the same event at different times. This is because differences and path lengths cause light from the event to take different times to get to Earth. Shorter path lengths take less time, but often go through denser material, which slows the light down. This competition between the two effects can help us determine distances to galaxies and thus the size of the universe.\\
	
	\noindent Recall the substructure problem, where we found that clusters and galaxies should look roughly the same (albeit on different size scales). There are two ways to explain this:
	\begin{enumerate}
		\item The Cosmological Model is wrong. Perhaps the assumptions we are using are simply not valid to explain how the universe works.
		\item Dark Matter Satellites. Perhaps there are enough satellites of galaxies, but they are composed mostly of dark matter. Perhaps a supernova blew all of the gas and dust out of them and left only dark matter. In this case, gravitational lensing should be able to detect these. This is a relatively new area of research.
	\end{enumerate}
	\subsubsection{Candidates for Dark Matter: Machos and Wimps}
	Dark matter is not part of the standard model of particle physics. That being said, if neutrinos have even a little bit of mass, they could partially explain dark matter, but they cannot be the entire solution. One might ask whether or not we are even able to see all the ordinary matter that exists. There was indeed a debate in science years ago about whether or not dark matter was baryonic (made of protons, neutrons, etc.) or not. Obviously by Occam's razor, we'd prefer that dark matter be baryonic so that we don't have to introduce an exotic particle to the theory.
	\paragraph{The Search for Baryonic Dark Matter} Scientists have found warm intergalactic gas that accounted for some missing mass. It was detected by X-Ray telescopes and was not visible by conventional means, but it was not nearly enough to account for the missing mass. It's rather difficult to hide baryonic gas since they will emit and absorb light to a level we can detect. 
	\paragraph{MACHOs} The only other option for hidden baryonic matter would be in compact objects like white dwarfs, neutron stars, brown dwarfs, or black holes. These objects are often called MACHOs, which stands for \textbf{M}\textbf{A}ssive \textbf{C}ompact \textbf{H}alo \textbf{O}bjects. For comparison, a white dwarf is about the size of Earth, despite having the mass of the sun. Neutron stars are also much more dense. However, there is not enough of these exotic objects to account for the dark matter. Similarly, we can detect brown dwarfs and the like, and they cannot account for the dark matter, either. Black Holes are much more difficult to detect since they themselves don't give off any radiation. Our only effective way to detect black holes is through a type of gravitational lensing called \textbf{microlensing}. They are such effective gravitational lenses, they brighten the objects the lens. After much study, though, it has been found that black holes, too, are unable to account for the dark matter.
	\paragraph{WIMPs} An appealing solution to the problem is the existence of WIMPs (\textbf{W}eakly \textbf{I}nteracting \textbf{M}assive \textbf{P}articles). We say they are ``weakly'' interacting since they interact \emph{only} through the weak nuclear force (i.e. they \emph{don't} interact through the electromagnetic force, and thus, light). An alternative to WIMPs are massive neutrinos (``Hot Dark Matter''), but this idea has grown out of favor.\\
	
	\noindent Scientists are currently conducting experiments underground to detect WIMPs by using Earth's natural shielding to prevent cosmic rays and other radiation from interfering with their experiments. So far, none have been discovered.
	
	\paragraph{Neutrinos} We know that neutrinos are very numerous (several hundred per cubic centimeter). If they don't have mass, they don't oscillate between different types of neutrinos. However, we notice that they in fact do oscillate between different types, so they do have mass, but how much mass, we are not certain. However, we have determined an upper bound on the mass of a neutrino which forbids them from being a sufficient explanation for the dark matter problem.
	\paragraph{Temperatures of Dark Matter} Cold dark matter would not be moving very fast, which allows for sufficient ``clumping'' of mass. Warm and hot dark matter move much faster, which would result in a smoother universe, which is not what we observe, so we tend to favor cold dark matter theories.
	\subsubsection{Another Option: A Misunderstanding of Gravity (MOND)} A theory that was popular in the eighties and nineties is MOND: \textbf{MO}dified \textbf{N}ewtonian \textbf{D}ynamics. This theory was a replacement of Newton's gravity that behaved differently in weak gravitational fields. It was a very elegant field that was proven wrong by experiment in an almost comically quick fashion. The basic idea behind MOND was that in the weak field, gravitational force is proportional to $R^{-1}$ rather than Newton's normal $R^{-2}$. Unfortunately, this didn't work out.\\
	
	\noindent Various examples, most notably the Bullet Cluster, shatter these theories by showing the dark matter mass not interacting in a cluster collision while interacting gases lingered in between the two resultant clusters. (Look up images and videos online to see how this works!)
	
	\section{Quasars, Active Galaxies, and Gamma Ray Bursts}
	An \textbf{Active Galactic Nucleus (AGN)} is the center of an ``active'' galaxy, that is, a galactic center with exceptionally bright and compact nuclear regions. The source of energy in an AGN is gravity, quite often in the form of a supermassive black hole. Things like blazars, QSO's, quasars, and radio galaxies are also common examples of AGN phenomena. These names are often annoying and pedantic, but we're stuck with them now.
	\subsection{Quasars}
	Quasars were first discovered in 1940 as powerful radio sources, though now we know that about 90\% of them are not very radio loud. They were objects that appeared to be stars whose spectra showed emission lines, which is quite strange, as most galaxies show absorption lines. Adding to the mystery was how these wavelengths were severely red-shifted. If these redshifts were in fact due to universal expansion, quasars had to be thousands of times brighter than the Milky Way.\\
	
	\noindent In 1963, M. Schmidt found Hydrogen Ballmer lines with redshifts $z=.158$, confirming that the mysterious objects were indeed extragalactic. It was later found that, though they were very faint, each quasar resided in a host galaxy. If we discard the spectrum of the quasar, we recover the normal spectra of galaxies.\\
	
	\noindent Since their discovery, we've discovered quasars up to $z=6$. The presence of quasars peaked at about 2 billion years after the big bang, and declined sharply ever since. They aren't made much any more, likely due to the dramatically slowed-down star formation rate.
	
	\subsubsection{Variability of a Quasar} Quasars are are very variable. That is, the flux from a quasar changes on a day-to-day basis! This means that they must be very small objects, so that different parts of it can ``communicate'' with itself so it gives off a more or less uniform signal. This is because the relative time scale for changes must be longer than $\tau\sim D/c$ where $D$ is the characteristic size of the object. This reflects the fact that no signal could travel from one end of the object to the other in any time less than $\tau$.
	\subsubsection{Power Source of Quasars}
	In order to be this small and yet so luminous, the energy source driving it must be extremely efficient. Nuclear fusion can no longer account for this, so gravity must be powering these objects.
	\subsubsection{Seyferts and Radio Galaxies}
	\textbf{Seyferts} are dim, radio-quiet quasars that reside primarily in spirals. \textbf{Radio Galaxies} are large galaxies with a quasar at its center that is loud in radio. A common feature of radio galaxies are powerful jets that are collimated over distances much larger than the size of their host galaxies. These jets typically only last a few hundred millions years.
	\subsubsection{Emission Mechanisms}
	AGNs use two types of emission mechanisms: Blackbody radiation and synchrotron radiation. As we have studied before, the intensity of blackbody radiation will always peak at a certain wavelength and decline in either direction thereafter. Synchrotron radiation, however, has a linear relationship between intensity and frequency.\\
	
	\noindent Gas spirals in towards a black hole, forming an accretion disk. The gas gets very hot and begins radiating black body radiation. Some particles are launched on the jets at relativistic speeds from the poles of the black holes. These jets are sometimes far more powerful than anything in the galaxy.
	\subsection{Unified Model for AGN}
	It has been asked whether or not all the different types of AGNs are completely different. The Unified MOdel of AGN posits that they are all indeed the same. The effects we observe differ only due to inclination effects. The standard AGN would then consist of a black hole, accretion disk, and jets. Depending on what angle we view the AGN, we observe different amounts of radio emission, jets, etc. However, some AGNs change in type from year to year, implying that inclination effects are not all that define how we see the AGN.\\
	
	\noindent It is becoming evident that probably all galaxies contain a super-massive black hole. They are only active, then, for short periods of time. The black hole in the galaxy seems to exert a lot of control over the evolution of its host galaxy. As we know, the Milk Way itself hosts an extraordinarily dim, super-massive black hole at its center. This has been observed by noting the motions of stars very near the center of the galaxy.
	\subsection{Gamma Ray Bursters (GRBs)}
	GRBs are super-energetic \emph{flashes} of gamma rays, not unlike what happens when nuclear weapons are detonated in the atmosphere. They come in two types:
	\begin{enumerate}
		\item Long-Duration (2-100 seconds)
		\item Short-Duration (.01-2 seconds)
	\end{enumerate}
	The distribution of GRBs is isotropic, suggesting that they are not in the disk of the Milky Way (we would expect them to be in the disk if they were nearby since that is where most objects in the Milky Way are located). Their nature remained elusive until 1997, when the first afterglow was detected. Absorption lines were measured with $z=0.835$: the GRB was 7 billion light years away! Such afterglows are typically found in the outskirts of galaxies. Thus, we surmise they are related to some dramatic event in a massive star's life.
	\subsubsection{Source of GRBs}
	As it turns out, GRBs are powerful jets from dying stars. They are the result of \textbf{hypernovae}, which are extremely energetic supernovae. The event, which is a collapsar, emits an extremely powerful jet that we see only if it is pointed in our direction.
\end{document}







